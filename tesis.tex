\documentclass[letterpaper,12pt]{book}
%Interlineado 1.5
%\renewcommand{\baselinestretch}{1.5}
\usepackage[utf8]{inputenc}
%Para evitar problemas de compatibilidad con el package tikz
\usepackage[spanish,es-noshorthands,es-tabla]{babel}
%\usepackage{subfigure}
%\usepackage[spanish,onelanguage]{algorithm}
%\usepackage{algorithmic}
\usepackage{algpseudocode}
\usepackage[spanish,onelanguage,ruled,linesnumbered]{algorithm2e} %ruled, vlined
\usepackage{listings}
\usepackage{amsthm}
\usepackage{textcomp}
\usepackage{multicol}
\usepackage{float}
\usepackage{url}
\usepackage{enumerate}
\usepackage{enumitem}
\usepackage{newlfont}
\usepackage{psfrag}
\usepackage{charter}
\usepackage{setspace}
\usepackage{longtable}
%\usepackage{subfigure}
\usepackage[dvips]{epsfig}
\usepackage[centertags]{amsmath}
\usepackage{newlfont}
\usepackage{psfrag}
\usepackage{textcomp}
\usepackage{framed}
\usepackage{verbatim}
\usepackage{multirow}
\usepackage{graphicx}
\usepackage[font=footnotesize]{caption}
\usepackage[font=scriptsize]{subcaption}
\usepackage{appendix}
%\usepackage{biblatex}

\usepackage[table]{xcolor}

\usepackage{fancyhdr}
\pagestyle{fancy}
\thispagestyle{plain}
\fancyhead[H]{} %header vacio
\renewcommand{\headrulewidth}{0pt}


%fonts matematicos
\usepackage{amsfonts}
\usepackage{amssymb}
\usepackage{amsmath}

%Diagramas
\usepackage{tikz}
\usetikzlibrary{positioning,calc,arrows,snakes,patterns}

%links
\usepackage{color}
 \usepackage{hyperref}
 \hypersetup{
     colorlinks=true,
     linktoc=section,
     linkcolor=red,
     citecolor=blue,
 }


\usepackage[top=2.5cm,bottom=2.5cm,left=4cm,right=2.5cm,asymmetric]{geometry}

\setlength{\parindent}{0pt}

%Profundidad Tabla de Contenidos
\setcounter{tocdepth}{4}
\setcounter{secnumdepth}{4}

% CAPITULO 1 NOMBRE
\usepackage{titletoc}     % http://ctan.org/pkg/titletoc
\titlecontents{chapter}   % <section-type>
  [0pt]% <left>
  {\addvspace{1em}}% <above-code>
  {\bfseries\chaptername\ \thecontentslabel\quad}% <numbered-entry-format>
  {}% <numberless-entry-format>
  {\bfseries\hfill\contentspage}% <filler-page-format>

\usepackage{apacite}

\renewcommand{\baselinestretch}{1.3}

\begin{document}
  \renewcommand{\BOthers}[1]{et al.\hbox{}}

  %Cambio de titulos
  \renewcommand*\contentsname{Índice General}
  \renewcommand*\listfigurename{Índice de Figuras}
  \renewcommand*\listtablename{Índice de Tablas}
  \renewcommand*\listalgorithmcfname{Índice de Algoritmos}
  %\renewcommand\spanishtablename{Tabla}
  \renewcommand{\appendixtocname}{Apéndices}
  \renewcommand{\appendixpagename}{Apéndices}

  %Definiciones
  \newcounter{defctr}
  \newenvironment{definition}{
    \refstepcounter{defctr}
    \begin{framed}
    \textbf{Definición \thedefctr}
    \newline
  }{\end{framed} \par \bigskip }
  \numberwithin{defctr}{chapter}

  \definecolor{light-gray}{gray}{0.95}
  \definecolor{light-gray-2}{gray}{0.8}


  % Estructura de contenidos para presentar
  % el trabajo de titulo

  \let\cleardoublepage\clearpage

  \thispagestyle{empty}
\begin{titlepage}

\begin{center}

\vspace*{-1in}
\begin{figure}[htb]
  \begin{flushleft}
    \includegraphics[width=10cm]{./figuras/logo_informatica.jpg}
  \end{flushleft}
\end{figure}

{\setstretch{1.5}
\vspace*{4cm}
\textbf{ \large{TITULO DE LA TESIS}} \\
\vspace*{2.0in}
}

\normalsize{por} \\
\vspace*{0.1in}
\normalsize{NOMBRE1 NOMBRE2 APELLIDO1 APELLIDO2} \\
\vspace*{1.3in}
\normalsize{Trabajo de Título presentado a la  \\ Facultad de Ingeniería de la Universidad Católica de Temuco  \\ Para Optar al Título de Ingeniero Civil en Informática \\}
\vspace*{0.5in}
\normalsize{- Temuco, MES AÑO -}
\end{center}

\end{titlepage}

  %\input{comision.tex}
  
\thispagestyle{empty}
\begin{titlepage}

{\setstretch{1.0}
\begin{center}
\vspace*{-1in}
\begin{figure}[htb]
\begin{center}
\includegraphics[width=8cm]{./figuras/logo_informatica.jpg}
\end{center}
\end{figure}

\vspace*{0.2in}
\textbf{ \normalsize{INFORME TRABAJO DE TÍTULO}} \\
\vspace*{0.2in}
\begin{flushleft}
\textbf{ \normalsize{TÍTULO:}} \hspace*{0.2in} \textbf{ \textit{ \normalsize{``TITULO DE LA TESIS''.}}}
\end{flushleft}
\vspace*{0.1in}
\begin{flushleft}
\textbf{ \normalsize{ALUMNO:}} \hspace*{0.2in} \textbf{ \normalsize{NOMBRE1 NOMBRE2 APELLIDO1 APELLIDO2}}
\end{flushleft}
\vspace*{0.3in}

\begin{flushleft}

\hspace{1cm} En mi calidad de Profesor Guía y luego de haber analizado y revisado el informe de Trabajo de Título puedo concluir lo siguiente:

\begin{list}{$\bullet$}{}
  \item Reportes del profesor
\end{list}

\hspace{1cm} De acuerdo a las previas observaciones califico el presente informe con \textbf {nota 7.0 (siete coma cero)}.
\end{flushleft}

\vspace*{0.5in}

\begin{flushright}
 \rule{65mm}{0.2mm}\\
\end{flushright}
\vspace*{-0.1in}
 \hspace*{3.5in} \textbf{Nombre Profesor Guia} \\
 \hspace*{3.5in} \textbf{Profesor Guía}

\vspace*{0.6in}

\begin{flushleft}
  Temuco, DIA de MES de AÑO.
\end{flushleft}

\end{center}
}
\end{titlepage}

  
\thispagestyle{empty}
\begin{titlepage}

{\setstretch{1.0}
\begin{center}
\vspace*{-1in}
\begin{figure}[htb]
\begin{center}
\includegraphics[width=8cm]{./figuras/logo_informatica.jpg}
\end{center}
\end{figure}

\vspace*{0.2in}
\textbf{ \normalsize{INFORME TRABAJO DE TÍTULO}} \\
\vspace*{0.2in}
\begin{flushleft}
\textbf{ \normalsize{TÍTULO:}} \hspace*{0.2in} \textbf{ \textit{ \normalsize{``TITULO DE LA TESIS''.}}}
\end{flushleft}
\vspace*{0.1in}
\begin{flushleft}
\textbf{ \normalsize{ALUMNO:}} \hspace*{0.2in} \textbf{ \normalsize{NOMBRE1 NOMBRE2 APELLIDO1 APELLIDO2}}
\end{flushleft}
\vspace*{0.3in}

\begin{flushleft}

\hspace{1cm} En mi condición de Profesor Informante presento el informe de calificación del Trabajo de Título del alumno
  NOMBRE1 NOMBRE2 APELLIDO1 APELLIDO2 bajo las siguientes observaciones:

\begin{list}{$\bullet$}{}
  \item Lo que dice el profesor informante.
\end{list}

\hspace{1cm} De acuerdo a estas consideraciones califico el desarrollo de éste Trabajo de Título con \textbf {nota 7.0 (siete coma cero)}.
\end{flushleft}

\vspace*{0.5in}

\begin{flushright}
 \rule{65mm}{0.2mm}\\
\end{flushright}
\vspace*{-0.2in}
 \hspace*{3.5in} \textbf{Nombre Profesor Informante} \\
 \hspace*{3.5in} \textbf{Profesor Informante}

\vspace*{0.6in}

\begin{flushleft}
Temuco, DIA de MES de AÑO.
\end{flushleft}

\end{center}
}
\end{titlepage}

  \thispagestyle{empty}
\begin{titlepage}
  {\setstretch{1.0}
  \vspace*{6.0in}
  \begin{flushright}

     Aquí va la dedicatoria

  \end{flushright}
  }
\end{titlepage}

  \thispagestyle{empty}
\begin{titlepage}
\begin{center}
\section*{AGRADECIMIENTOS}
\end{center}

%\newgeometry{margin=1cm}

\begin{flushleft}

  Aquí van los agradecimientos

\end{flushleft}
\end{titlepage}



  % Indice
  % Numeracion romana
  \pagenumbering{Roman}

  \tableofcontents
  \addcontentsline{toc}{chapter}{Índice General}
  \cleardoublepage

  \phantomsection
  \addcontentsline{toc}{chapter}{Índice de Figuras}
  \listoffigures
  \cleardoublepage

  \phantomsection
  \addcontentsline{toc}{chapter}{Índice de Tablas}
  \listoftables
  \cleardoublepage

  \phantomsection
  \addcontentsline{toc}{chapter}{Índice de Algoritmos}
  \listofalgorithms
  \cleardoublepage

  % Resumen
  \input{./extras/resumen.tex}
  \cleardoublepage


  % Capitulos del trabajo
  % Uso de numeracion Arabica
  \pagenumbering{arabic}

  
%% CAPÍTULO 1 %%
\chapter{Introducción}

  %% SECCIÓN %%
  \section{Introducción a X}

  Lorem ipsum dolor sit amet, consectetur adipiscing elit. Quisque cursus imperdiet
  pharetra. Donec aliquet sem non leo bibendum cursus. In dictum vel mauris non
  laoreet. Integer in metus magna. Fusce tempus non magna cursus aliquet. Donec
  vitae lobortis felis. Duis tortor sapien, egestas sed leo vitae, sodales hendrerit
  lorem. Nullam sed iaculis lectus, sit amet efficitur purus. In id orci eget lacus
  feugiat vel sed magna.

  Quisque sed lorem et mi pulvinar sollicitudin eleifend non ante. Nullam pretium
  dignissim orci id posuere. Maecenas vehicula ex a enim laoreet, posuere consectetur
  magna rutrum. Suspendisse mi odio, semper quis fringilla sit amet, aliquet ut eros.

  \begin{figure}[!htb]
    \begin{center}
    	\includegraphics[width=15cm]{./figuras/wavelets.jpg}
    	\caption{Ejemplos de figura}
    \end{center}
    \label{fig:figura}
  \end{figure}

  %% SECCIÓN %%
  \section{Propuesta de Tesis}

  Lorem ipsum dolor sit amet, consectetur adipiscing elit. Quisque cursus imperdiet
  pharetra. Donec aliquet sem non leo bibendum cursus. In dictum vel mauris non
  laoreet. Integer in metus magna. Fusce tempus non magna cursus aliquet. Donec
  vitae lobortis felis. Duis tortor sapien, egestas sed leo vitae, sodales hendrerit
  lorem. Nullam sed iaculis lectus, sit amet efficitur purus. In id orci eget lacus
  feugiat vel sed magna.


  %% SECCIÓN %%
  \section{Contribución del Trabajo}

    sit amet efficitur purus. In id orci eget lacus feugiat vel sed magna.

    \subsection{Aportes}

      Lorem ipsum dolor sit amet, consectetur adipiscing elit. Quisque cursus imperdiet
      pharetra. Donec aliquet sem non leo bibendum cursus. In dictum vel mauris non
      laoreet.

    \subsection{Resultados Esperados}

      Lorem ipsum dolor sit amet, consectetur adipiscing elit. Quisque cursus imperdiet
      pharetra. Donec aliquet sem non leo bibendum cursus. In dictum vel mauris non
      laoreet.

  %% SECCIÓN %%
  \section{Estructura del Trabajo}

    Este trabajo de título se dividirá en los siguientes 4 capítulos, los cuales se describen a continuación:

    \begin{itemize}
      \item \textbf{Capítulo 1: Introducción} \\

      Lorem ipsum dolor sit amet, consectetur adipiscing elit. Quisque cursus imperdiet
      pharetra. Donec aliquet sem non leo bibendum cursus. In dictum vel mauris non
      laoreet.

      \item \textbf{Capítulo 2: Estado del Arte} \\

      Lorem ipsum dolor sit amet, consectetur adipiscing elit. Quisque cursus imperdiet
      pharetra. Donec aliquet sem non leo bibendum cursus. In dictum vel mauris non
      laoreet.

      \item \textbf{Capítulo 3: Materiales y Métodos} \\

      Lorem ipsum dolor sit amet, consectetur adipiscing elit. Quisque cursus imperdiet
      pharetra. Donec aliquet sem non leo bibendum cursus. In dictum vel mauris non
      laoreet.

      \item \textbf{Capítulo 4: Resultados Experimentales} \\

      Lorem ipsum dolor sit amet, consectetur adipiscing elit. Quisque cursus imperdiet
      pharetra. Donec aliquet sem non leo bibendum cursus. In dictum vel mauris non
      laoreet.

      \item \textbf{Capítulo 5: Conclusiones y Trabajo a Futuro} \\

      Lorem ipsum dolor sit amet, consectetur adipiscing elit. Quisque cursus imperdiet
      pharetra. Donec aliquet sem non leo bibendum cursus. In dictum vel mauris non
      laoreet.

    \end{itemize}


  % Bibliografia - Referencias
  \cleardoublepage
  \addcontentsline{toc}{chapter}{Bibliografía}
  \bibliographystyle{apacitex}
  \bibliography{referencias}

  % Apendices
  \cleardoublepage
\addappheadtotoc
\appendixpage
\appendix
  \chapter{Resultados Experimentales Completos} \label{apenA}
    \section{TIPO}
      \clearpage
      \subsection{DATOS X}
	     \subsubsection{METODO Y}
	\clearpage


\end{document}
